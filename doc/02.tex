\documentclass{article}

\usepackage{amssymb, amsmath}
\usepackage{alltt}
\usepackage{pslatex}
\usepackage{epigraph}
\usepackage{verbatim}
\usepackage{latexsym}
\usepackage{array}
\usepackage{comment}
\usepackage{makeidx}
\usepackage{listings}
\usepackage{indentfirst}
\usepackage{verbatim}
\usepackage{color}
\usepackage{url}
\usepackage{xspace}
\usepackage{hyperref}
\usepackage{stmaryrd}
\usepackage{amsmath, amsthm, amssymb}
\usepackage{graphicx}
\usepackage{euscript}
\usepackage{mathtools}
\usepackage{mathrsfs}
\usepackage{multirow,bigdelim}

\makeatletter

\makeatother

\definecolor{shadecolor}{gray}{1.00}
\definecolor{darkgray}{gray}{0.30}

\def\transarrow{\xrightarrow}
\newcommand{\setarrow}[1]{\def\transarrow{#1}}

\def\padding{\phantom{X}}
\newcommand{\setpadding}[1]{\def\padding{#1}}

\newcommand{\trule}[2]{\frac{#1}{#2}}
\newcommand{\crule}[3]{\frac{#1}{#2},\;{#3}}
\newcommand{\withenv}[2]{{#1}\vdash{#2}}
\newcommand{\trans}[3]{{#1}\transarrow{\padding#2\padding}{#3}}
\newcommand{\ctrans}[4]{{#1}\transarrow{\padding#2\padding}{#3},\;{#4}}
\newcommand{\llang}[1]{\mbox{\lstinline[mathescape]|#1|}}
\newcommand{\pair}[2]{\inbr{{#1}\mid{#2}}}
\newcommand{\inbr}[1]{\left<{#1}\right>}
\newcommand{\highlight}[1]{\color{red}{#1}}
\newcommand{\ruleno}[1]{\eqno[\scriptsize\textsc{#1}]}
\newcommand{\rulename}[1]{\textsc{#1}}
\newcommand{\inmath}[1]{\mbox{$#1$}}
\newcommand{\lfp}[1]{fix_{#1}}
\newcommand{\gfp}[1]{Fix_{#1}}
\newcommand{\vsep}{\vspace{-2mm}}
\newcommand{\supp}[1]{\scriptsize{#1}}
\newcommand{\sembr}[1]{\llbracket{#1}\rrbracket}
\newcommand{\cd}[1]{\texttt{#1}}
\newcommand{\free}[1]{\boxed{#1}}
\newcommand{\binds}{\;\mapsto\;}
\newcommand{\dbi}[1]{\mbox{\bf{#1}}}
\newcommand{\sv}[1]{\mbox{\textbf{#1}}}
\newcommand{\bnd}[2]{{#1}\mkern-9mu\binds\mkern-9mu{#2}}

\newcommand{\meta}[1]{{\mathcal{#1}}}
\renewcommand{\emptyset}{\varnothing}

\definecolor{light-gray}{gray}{0.90}
\newcommand{\graybox}[1]{\colorbox{light-gray}{#1}}

\lstdefinelanguage{ocaml}{
keywords={let, begin, end, in, match, type, and, fun, 
function, try, with, class, object, method, of, rec, repeat, until,
while, not, do, done, as, val, inherit, module, sig, @type, struct, 
if, then, else, open, virtual, new, fresh},
sensitive=true,
%basicstyle=\small,
commentstyle=\scriptsize\rmfamily,
keywordstyle=\ttfamily\bfseries,
identifierstyle=\ttfamily,
basewidth={0.5em,0.5em},
columns=fixed,
fontadjust=true,
literate={fun}{{$\lambda$}}1 {->}{{$\to$}}3 {===}{{$\equiv$}}1 {=/=}{{$\not\equiv$}}1 {|>}{{$\triangleright$}}3 {\&\&\&}{{$\wedge$}}2 {|||}{{$\vee$}}2 {^}{{$\uparrow$}}1,
morecomment=[s]{(*}{*)}
}

\lstset{
mathescape=true,
%basicstyle=\small,
identifierstyle=\ttfamily,
keywordstyle=\bfseries,
commentstyle=\scriptsize\rmfamily,
basewidth={0.5em,0.5em},
fontadjust=true,
escapechar=!,
language=ocaml
}

\sloppy

\newcommand{\ocaml}{\texttt{OCaml}\xspace}

\theoremstyle{definition}

\title{Statements, Stack Machine, Stack Machine Compiler\\
  (the first draft)
}
\date{\vspace{-5ex}}

\author{Dmitry Boulytchev}

\begin{document}

\maketitle

\section{Statements}

More interesting language~--- a language of simple statements:

$$
\begin{array}{rcl}  
  \mathscr S & = & \mathscr X \mbox{\lstinline|:=|} \;\mathscr E \\
             &   & \mbox{\lstinline|read (|} \mathscr X \mbox{\lstinline|)|} \\
             &   & \mbox{\lstinline|write (|} \mathscr E \mbox{\lstinline|)|} \\
             &   & \mathscr S \mbox{\lstinline|;|} \mathscr S
\end{array}
$$

Here $\mathscr E, \mathscr X$ stand for the sets of expressions and variables, as in the previous lecture.

Again, we define the semantics for this language 

$$
\sembr{\bullet}_{\mathscr S} : \mathscr S \mapsto \mathbb Z^* \to \mathbb Z^*
$$

with the semantic domain of partial functions from integer strings to integer strings. This time we will
use \emph{big-step operational semantics}: we define a ternary relation ``$\Rightarrow$''

$$
\Rightarrow \subseteq \mathscr C \times \mathscr S \times \mathscr C
$$

where $\mathscr C = \Sigma \times \mathbb Z^* \times \mathbb Z^*$~--- a set of all configurations during a
program execution. We will write $c_1\xRightarrow{S}c_2$ instead of $(c_1, S, c_2)\in\Rightarrow$ and informally
interpret the former as ``the execution of a statement $S$ in a configuration $c_1$ completes with the configuration
$c_2$''. The components of a configuration are state, which binds (some) variables to their values, and input and
output streams, represented as (finite) strings of integers.

The relation ``$\Rightarrow$'' is defined by the following deductive system (see Fig.~\ref{bs_stmt}). The first
three rules are \emph{axioms} as they do not have any premises. Note, according to these rules sometimes a program
cannot do a step in a given configuration: a value of an expression can be undefined in a given state in rules
$\rulename{Assign}$ and $\rulename{Write}$, and there can be no input value in rule $\rulename{Read}$. This style of
a semantics description is called big-step operational semantics, since the results of a computation are
immediately observable at the right hand side of ``$\Rightarrow$'' and, thus, the computation is performed in
a single ``big'' step. And, again, this style of a semantic description can be used to easily implement a
reference interpreter.

With the relation ``$\Rightarrow$'' defined we can abbreviate the ``surface'' semantics for the language of statements:

\setarrow{\xRightarrow}

\[
\forall S\in\mathscr S,\,\forall i\in\mathbb Z^*\;:\;\sembr{S}_{\mathscr S} i = o \Leftrightarrow \trans{\inbr{\bot, i, \epsilon}}{S}{\inbr{\_, \_, o}}
\]


\begin{figure}[t]
\[\trans{\inbr{s, i, o}}{\llang{x := $\;\;e$}}{\inbr{s[x\gets\sembr{e}_{\mathscr E}\;s], i, o}}\ruleno{Assign}\]
\[\trans{\inbr{s, z\llang{::}i, o}}{\llang{read ($x$)}}{\inbr{s[x\gets z], i, o}}\ruleno{Read}\]
\[\trans{\inbr{s, i, o}}{\llang{write ($e$)}}{\inbr{s, i, o \llang{@} \sembr{e}_{\mathscr E}\;s}}\ruleno{Write}\]
\[\trule{\trans{c_1}{S_1}{c^\prime},\;\trans{c^\prime}{S_2}{c_2}}{\trans{c_1}{S_1\llang{;}S_2}{c_2}}\ruleno{Seq}\]
\caption{Big-step operational semantics for statements}
\label{bs_stmt}
\end{figure}

\section{Stack Machine}

Stack machine is a simple abstract computational device, which can be used as a convenient model to constructively describe
the compilation process.

In short, stack machine operates on the same configurations, as the language of statements, plus a stack of integers. The
computation, performed by the stack machine, is controlled by a program, which is described as follows:

\[
\begin{array}{rcl}
  \mathscr I & = & \llang{BINOP $\;\otimes$} \\
             &   & \llang{CONST $\;\mathbb N$} \\
             &   & \llang{READ} \\
             &   & \llang{WRITE} \\
             &   & \llang{LD $\;\mathscr X$} \\
             &   & \llang{ST $\;\mathscr X$} \\
  \mathscr P & = & \epsilon \\
             &   & \mathscr I\mathscr P
\end{array}
\]

Here the syntax category $\mathscr I$ stands for \emph{instructions}, $\mathscr P$~--- for \emph{programs}; thus, a program is a finite
string of instructions.

The semantics of stack machine program can be described, again, in the form of big-step operational semantics. This time the set of
stack machine configurations is

\[
\mathscr C_{SM} = \mathbb Z^* \times \mathscr C
\]

where the first component is a stack, and the second~--- a configuration as in the semantics of statement language. The rules are shown on Fig.~\ref{bs_sm}; note,
now we have one axiom and six inference rules (one per instruction).

As for the statement, with the aid of the relation ``$\Rightarrow$'' we can define the surface semantics of stack machine:

\[
\forall p\in\mathscr P,\,\forall i\in\mathbb Z^*\;:\;\sembr{p}_{SM}\;i=o\Leftrightarrow\trans{\inbr{\epsilon, \inbr{\bot, i, \epsilon}}}{p}{\inbr{\_, \inbr{\_, \_, o}}}
\]

\begin{figure}[t]
  \[\trans{c}{\epsilon}{c}\ruleno{Stop$_{SM}$}\]
  \[\trule{\trans{\inbr{(x\oplus y)\llang{::}st, c}}{p}{c^\prime}}{\trans{\inbr{y\llang{::}x\llang{::}st, c}}{(\llang{BINOP $\;\otimes$})p}{c^\prime}}\ruleno{Binop$_{SM}$}\]
  \[\trule{\trans{\inbr{z\llang{::}st, c}}{p}{c^\prime}}{\trans{\inbr{st, c}}{(\llang{CONST $\;z$})p}{c^\prime}}\ruleno{Const$_{SM}$}\]
  \[\trule{\trans{\inbr{z\llang{::}st, \inbr{s, i, o}}}{p}{c^\prime}}{\trans{\inbr{st, \inbr{s, z\llang{::}i, o}}}{(\llang{READ})p}{c^\prime}}\ruleno{Read$_{SM}$}\]
  \[\trule{\trans{\inbr{st, \inbr{s, i, o\llang{@}z}}}{p}{c^\prime}}{\trans{\inbr{z\llang{::}st, \inbr{s, i, o}}}{(\llang{WRITE})p}{c^\prime}}\ruleno{Write$_{SM}$}\]
  \[\trule{\trans{\inbr{(s\;x)\llang{::}st, \inbr{s, i, o}}}{p}{c^\prime}}{\trans{\inbr{st, \inbr{s, i, o}}}{(\llang{LD $\;x$})p}{c^\prime}}\ruleno{LD$_{SM}$}\]
  \[\trule{\trans{\inbr{st, \inbr{s[x\gets z], i, o}}}{p}{c^\prime}}{\trans{\inbr{z\llang{::}st, \inbr{s, i, o}}}{(\llang{ST $\;x$})p}{c^\prime}}\ruleno{ST$_{SM}$}\]
  \caption{Big-step operational semantics for stack machine}
  \label{bs_sm}
\end{figure}

\section{A Compiler for the Stack Machine}

A compiler of the statement language into the stack machine is a total mapping

\[
\sembr{\bullet}_{comp} : \mathscr S \mapsto \mathscr P
\]

We can describe the compiler in the form of denotational semantics for the source language. In fact, we can treat the compiler as a \emph{static} semantics, which
maps each program into its stack machine equivalent.

As the source language consists of two syntactic categories (expressions and statments), the compiler has to be ``bootstrapped'' from the compiler for expressions
$\sembr{\bullet}^{\mathscr E}_{comp}$:

\[
\begin{array}{rcl}
  \sembr{x}^{\mathscr E}_{comp}&=&\llang{[LD $\;x$]}\\
  \sembr{n}^{\mathscr E}_{comp}&=&\llang{[CONST $\;n$]}\\
  \sembr{A\otimes B}^{\mathscr E}_{comp}&=&\sembr{A}^{\mathscr E}_{comp}\llang{@}\sembr{B}^{\mathscr E}_{comp}\llang{@}[\llang{BINOP $\;\otimes$}]
\end{array}
\]

And now the main dish:

\[
\begin{array}{rcl}
  \sembr{\llang{$x$ := $e$}}_{comp}&=&\sembr{e}^{\mathscr E}_{comp}\llang{@}\llang{[ST $\;x$]}\\
  \sembr{\llang{read ($x$)}}_{comp}&=&\llang{[READ; ST $\;x$]}\\
  \sembr{\llang{write ($e$)}}_{comp}&=&\sembr{e}^{\mathscr E}_{comp}\llang{@}\llang{[WRITE]}\\
  \sembr{\llang{$S_1$;$\;S_2$}}_{comp}&=&\sembr{S_1}_{comp}\llang{@}\sembr{S_2}_{comp}
\end{array}
\]

\end{document}

%%% Local Variables:
%%% mode: xelatex
%%% TeX-master: t
%%% End:
